\documentclass[10pt]{article}
\usepackage[english]{babel}

\usepackage{braket}
\usepackage{breqn}
\usepackage{graphicx}
\usepackage{color}
\usepackage{cite}
\usepackage{hyperref}
\newcommand{\R}{\mathbb{R}}
\newcommand*{\refequa}[1]{(\ref{#1})}
\newcommand*{\ham}{\mathcal{H}}
\newcommand*{\partialD}[2] {\frac{\partial #1}{\partial #2}}
\newcommand*{\sumproper}[2] {\underset{#1}{\overset{#2}{\sum}}}
\newcommand*{\mean}[1]{\langle {#1}\rangle}

\newcommand{\beq}{\begin{equation}}
\newcommand{\eeq}{\end{equation}}
\newcommand{\tobs}{t_{\rm obs}}
\newcommand{\gammaobs}{\gamma_{\mathrm{obs}}}
\newcommand{\Phit}{\tilde\Phi}
\newcommand{\PP}{\mathcal{P}}
\newcommand{\QQ}{\mathcal{Q}}

\newcommand{\LL}{ L_{\rm r} }  %% this is what used to be L'

\newcommand{\rlj}[1]{{\color{blue}#1}}
\newcommand{\jd}[1] {{\color{red}#1}}

\begin{document}
\section{Priors for parameters}
The best priors for the age independent rates we could find for SEAI8R are (in units of [1/day])\\
\\
$\gamma_{E}=1/2.72$   from exposed to activated $95\%$ CI:  $1/2.55$ $1/2.89$\\
\\
$\gamma_{A}= 1/3.12$   from activated to infected $95\%$ CI: $1/(2.08)$ $1/(4.16)$\\
\\
$\gamma_{Ia}= 1./7$           recovery rate of asymptomatic infectives (Can't really find a source for this beyond conventional wisdom)\\
\\
$\gamma_{Is}   = 1./4.82$  from symptomatic to hospitalised $95\%$ CI: $1/3.487$  $1/6.157$\\
\\
$\gamma_{Is}+\gamma_{Isp}=17.76$ from symptomatic to recovered, this value is very suspect and should be updated $95\%$ CI: $12.64$ $22.87$\\
\\
$\gamma_{Ih}+\gamma_{Ihp}= \ln(2)/10$ from hospital to recovered/leaving hospital (IQR $\ln(2)/7.0$ $\ln(2)/14.0$)\\
\\
$\gamma_{Ih} = 1/5.66$  from hospitalised to ICU $95\%$ CI: $1/4.18$ $1/7.14$\\
\\
$\gamma_{Ic} = 1/4  $ from ICU to recovered/leaving ICU IQR ($1/3$, $1/5$)\\
\\
$\gamma_{Ic} +\gamma_{Icp}  = 1/5.45$  from ICU to death $95\%$  CI:  $1/(2.20)$ $1/(7.65)$\\
\subsection{Sources}
These are the sources of some of the priors we use.\\
\\
\href{https://www.medrxiv.org/content/10.1101/2020.03.03.20029983v1}{Transmission interval estimates suggest pre-symptomatic spread of COVID-19} -source of why we think presymptomatic spreading exists and $\gamma_{E}$\\
\\
\href{https://www.medrxiv.org/content/10.1101/2020.04.01.20050138v1.full.pdf}{Epidemiological Characteristics of COVID-19; a Systemic Review and Meta-Analysis} source of $\gamma_{A}$, $\gamma_{Ia}$, $\gamma_{Is}$,$\gamma_{Ih}$,  $\gamma_{Ic}$ and
$\gamma_{Is}(1-hh)+\gamma_{Isp}$. This  $\gamma_{Is}$ probably isn't the one we should be using but it is the best we found for now.\\
\\
\href{https://www.medrxiv.org/content/10.1101/2020.04.08.20056861v1.full.pdf}{Epidemiological characteristics of COVID-19 cases in Italy and estimates of the reproductive numbers one month into the epidemic} This paper is one of our sources for the data in the accompanying ipython notebooks for hh,cc,mm and alpha.\\
\\
\href{https://www.icnarc.org/DataServices/Attachments/Download/76a7364b-4b76-ea11-9124-00505601089b}{ICNARC report on COVID-19 in critical care} for $\gamma I_c+gI_{cp}$\\
\\
\href{https://jamanetwork.com/journals/jama/fullarticle/2761044}{Clinical Characteristics of 138 Hospitalized Patients With 2019 Novel Coronavirus–Infected Pneumonia in Wuhan, China} for $\gamma_{Ih}+\gamma_{Ihp}$\\
\\
 \href{https://www.medrxiv.org/content/10.1101/2020.03.30.20047365v2.full.pdf}{Immunity paper}
 This seems to indicate roughly a third of mild cases develop a very weak immunity\\
\\
 \href{https://www.medrxiv.org/content/10.1101/2020.03.24.20043018v2.full.pdf}{Susceptibility}This indicates younger people are less susceptible.  Roughly by two thirds less susceptible and it more or less goes to a constant by age 30.
\end{document} 